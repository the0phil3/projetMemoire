\part*{Conclusion}
\addcontentsline{toc}{part}{Conclusion}
\markboth{Conclusion}{Conclusion} 
Mon rapport constitue un premier pas vers une étude compréhensive et raisonnée de la guerre du Rif. En ce qui concerne mes sources, ma méthodologie et mes résultats, mon travail a consisté à essayer de développer une méthode qui pourrait être utilisée pour tirer des conclusions pertinentes par rapport aux enjeux d’aujourd’hui avec les outils du moment.\\ 

Mes sources forment la colonne verticale de mon mini-mémoire et constitueront la base de mon mémoire final. Fondamentalement, je suis satisfait de leur capacité à se complémenter mutuellement. Cependant, je reconnais que l'utilisation de deux de mes trois sources primaires (les registres matricules et les JMOs) soulève des questions quant à leur caractère pratique. Les problèmes d’échantillonnage des registres matricules et l'entraînement des modèles qui vont les transcrire n’ont pas encore trouvé de réponse. Ils méritent que je m’y attarde au plus vite afin de mieux préparer mes démarches pour l'année prochaine. D’autre part, le fait que je n’ai pas encore eu le temps de vérifier l’état des JMOs de la guerre du Rif le place en tête de mes priorités. Une fois que ces deux points auront été réglés,  je pourrai procéder à mes recherches de manière méthodique.\\ 

Dans sa globalité, ma méthodologie prend en compte les complexités du terrain et tente d’y répondre  sous un maximum d’angles. Cela peut entraîner des contraintes de temps et des difficultés de collecte et le traitement de données mais c’ est nécessaire pour que je puisse garder un lien fort entre mes archives qualitatives et quantitatives. Le but de ce lien étant de ne pas se limiter à l’une de ces deux catégories.\\ 

Les résultats que j’ai pu montrer ont servi pour éclairer des détails intéressants sur le déroulement des combats. Il n’y avait que peu ou pas de valeurs aberrantes dans mes résultats, même en tenant compte de l’absence  d’un objet de comparaison pour certaines données. Cependant, je reconnais que l’extrapolation et le traitement de la base de données des décès auraient pu être plus approfondis.  Plus précisément, du côté du code, j'aurais pu rechercher davantage de relations entre les différentes caractéristiques des soldats morts  pendant cette guerre. J’en suis conscient et je sais que je dois encore effectuer un travail d'apprentissage des modèles de régression de \textbf{scikit-learn}et d’approfondissement de mes compétences avec \\textbf{pandas} en \textbf{python}.\\ 

