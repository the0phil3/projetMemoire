\part*{Introduction}
\addcontentsline{toc}{part}{Introduction}
\markboth{Introduction}{Introduction}
Force est de constater que le débat sur la mémoire des guerres mondiales et la colonisation est plus que jamais d’actualité sur la scène politique française. Si les historiens ont travaillé sur la question de la contribution de l'empire français aux deux guerres mondiales, des conflits de moindre ampleur sont restés largement oubliés. Les réponses à ces questions ne sont pas si évidentes, surtout si l'on tient compte des différentes guerres de l’histoire contemporaine française dans lesquelles les  troupes coloniales ont été engagées depuis la période napoléonienne. Il n'y a pas de consensus sur  la question des soldats morts pendant la guerre du Rif. L’historiographie tend vers une réponse qui place les pertes des soldats nord-africains et de la coloniale au-dessus de celles des soldats métropolitains.\footcites{schiavon2016} Mais cette thèse n’est pas hégémonique car la guerre a connu différentes phases avec des moyens et des concentrations de troupes très variés, et la participation des différentes composantes n'a jamais été quantifiée. L’histoire de cette guerre a souvent été oubliée mais elle suscite aujourd'hui un regain d'intérêt , probablement en raison de son caractère unique et prémonitoire.\footcites{marly2021} La guerre du Rif a été un précurseur des guerres post-coloniales car c'est  la première guerre du 20eme siècle où un peuple indigène a cherché  à se débarrasser de son colonisateur européen. C'est aussi  la première guerre moderne dans le sens où elle a intégré  l’aviation, les chars et l'utilisation d'armes  chimiques (de manière systématique  par l’armée espagnole). Elle peut donc  être considérée comme la première « opération extérieure »\footnote{Le terme contemporain désignant le déploiement de soldats français hors de la metropole.} de l’armée française. En fait, sa caractérisation fluctue entre celle d’une guerre de conquête coloniale, d’une guerre de décolonisation et d’une guerre post-coloniale. Si la guerre du Rif est riche d'enseignements sur l'histoire de l'armée française et son rapport au colonialisme, elle est également révélatrice des fondements de la société marocaine qui perdurent encore aujourd'hui.\\  

Les montagnes du Rif au Maroc ont une longue histoire de contestation sociale, qui remonte à avant l'arrivée du pouvoir français dans le pays. La guerre du Rif de 1921 à 1926 fait partie de l’histoire mouvementée de cette région. Le bilan anthropologique du Rif réalisé par Khalid Mouna expose la singularité du cas Rifain dans le paysage Marocain. Cependant une comparaison approfondie avec des études sociologiques d’autres régions berbères du Maghreb ne permet pas de distinguer spécifiquement ce cas. C’est en effet la situation agricole dans le Rif qui a façonné les relations entre ses habitants et son histoire.\footcites[101]{mouna2008} Historiquement surpeuplé et ayant des terres pauvres, différents sociologues, de Bourdieu à Jamous, se sont penchés sur la prétendue anarchie qui régit le Rif. Les mouvements sociaux  dans le Rif peuvent et doivent être placés dans un contexte régional. Le soulèvement de 1958 et le mouvement populaire du Rif en 2016-2017 montrent que les tensions entre le pouvoir central marocain et les habitants du Rif n’ont pas disparu avec la fin du protectorat. L'œuvre d’Ibn Khaldun, \underline{La Muqaddima}, affirme l’idée que la perspective historique est fondamentale afin de comprendre les changements que peut connaître une société.\footcites[103]{mouna2008}  Quand Mouna essaye d’appliquer ce qu’il qualifie d’une « démarche khaldounienne » à ces questions, nous pouvons constater la continuité du même rôle que le Rif joue à travers l’histoire. C’est dans cette veine  que Mouna explique les liens puissants entre le Rif et la production du kif (résine de cannabis).\footcites[105]{mouna2008} Un saint-homme, fondateur de la confrérie soufie Haddawa, Sidi Haddi s’est installé dans la tribu des Ketama dans le Rif central, au début 19eme siècle. Cette confrérie est la seule à prôner l'usage du cannabis dans un cadre islamique. Quand on sait que la quasi-totalité de la culture du cannabis au Maroc aujourd’hui se situe dans le Rif, les liens entre la guerre du Rif et la culture de contrebande et des hors-la-loi peuvent paraître plus clairs.\footcites{chouvy2007} D’autant plus que des historiens espagnols ont démontré à quel point le Rif et l'Espagne ont été complices dans le trafic d'armes et de contrebande tout au long de la guerre.\\


La guerre du Rif a commencé en 1921, quand les Espagnols ont été vaincus à la bataille d’Anoual par l’armée rifaine d’Abdelkrim. Le conflit a éclaté à la suite d’une nouvelle politique sur  le territoire de leurs comptoirs marocains : les militaires espagnols ayant décidé de quitter la plaine pour installer des postes dans le massif du Rif, territoire des rifains, un peuple farouchement indépendant et belliqueux.\footcites[32]{ayache1996} Après un début catastrophique pour l’armée espagnole, la situation se stabilise  jusqu'en 1925, lorsque les Rifains attaquent le territoire français dans la partie sud des montagnes du Rif. C’est lorsque Abdelkrim lance son offensive vers le nord du protectorat français, que la guerre du Rif commence réellement. Elle va durer de 1925 à 1926 pour les Français jusqu’à la reddition d' Abdelkrim en mai 1926. Il est utile pour pouvoir analyser  les archives de comprendre que durant  cette période, la guerre a connu trois phases  différentes : ce que les militaires appellent la période « héroïque » (avril à août 1925), la période de stabilisation du front (septembre 1925 à avril 1926), et la période de guerre totale (mars à mai 1926).\footcites[79]{schiavon2016} Le début de la guerre est appelé la période « héroïque » car l’armée française a subi ses plus grandes pertes pendant cette phase de la guerre. Elle est en sous-effectif et elle doit  manoeuvrer et combattre sur tout le territoire du protectorat pour contenir les débordements rifains. La période de stabilisation du front porte ce nom parce qu’il décrit ce qui s’est passé pendant que les Français préparaient une offensive majeure. La période de guerre totale correspond à l'arrivée du maréchal Pétain au pouvoir en France et à une augmentation considérable des troupes et des moyens matériels de l’armée française au Maroc. L’armée française subit très peu de pertes pendant cette offensive. Connaître ces trois périodes est utile car elles sont censées correspondre aux éléments chiffrées des pertes dans les bases de données des archives. \\


La littérature secondaire sur la guerre du Rif est étonnamment mince. La tendance la plus répandue dans  l’historiographie se concentre sur les effets de la guerre sur la politique espagnole. Cette tendance est prédominante en raison  de la position centrale que la guerre a joué dans l'ascension de Franco et la future guerre civile espagnole. Les historiographies anglaise et espagnole sont dominées par ce point de vue.\footcites{harris1925} En revanche, les historiographies  française et marocaine accordent plus d'importance  à l’étude du phénomène Rifain et à la République d’Abdelkrim. La littérature secondaire française se concentre sur le rôle de l’armée, des politiques et des grandes figures qui ont mené  la guerre (Lyautey et Pétain). Elle s'intéresse très sporadiquement à inscrire la guerre dans une logique coloniale en étudiant les différentes composantes de l’empire qui y ont participé. \\


Sur le plan numérique et quantitatif, il n’existe à ma connaissance qu’un seul article qui traite de la guerre du Rif en humanités numériques : le travail d’analyse des réseaux de Julián López sur la contrebande pendant la guerre du Rif.\footcites{lopez2016} En histoire militaire dans la période 14-18, Henri Gilles, Jean-Pascal Guironnet, et Antoine Parent ont réalisé une étude proche  de celle que j’envisage de faire sur la \underline{« Géographie économique des morts de 14-18 en France »}.  À partir d'une base de données du SHD de la Première guerre mondiale, ils ont essayé de dénombrer et de trier les individus Morts pour la France (MPF) par régions et de comparer les données MPF avec celles du recensement de 1911. Cela leur a permis d'extrapoler un calcul des décès en proportion des mobilisables par région.\footcites[521]{gilles2014} Ils ont ensuite construit un modèle pour expliquer  la différence en proportions de pertes de chaque région. Pour représenter cet indicateur, ils ont imaginé les deux effets qui pouvaient y contribuer : le taux de mobilisables et la densité de population. En multipliant ces deux facteurs, ils sont arrivés avec l’indicateur des mobilisables par surface :
$
\frac{mobilisables}{surface} = 
\frac{mobilisables}{population}* 
\frac{population}{surface}. 
$\footcites[523]{gilles2014}
Ils ont aussi essayé de créer un modèle dit « historique » qui prend en compte les variables géographiques (distance au front ou à la frontière) ou le taux de ruralité dans la région.\\


En contrepartie,dans \underline{Tous égaux devant « l’impôt du sang» ?} de la même revue, André Loez et Nicolas Mariot, replacent l’article précédent dans son contexte historiographique (par rapport à l’histoire quantitative et celle de 14-18) et contestent l’utilisation anachronique des régions choisies pour des données qui n’étaient pas dans ce format originel. L’autre problème souligné par Mariot et Loez est la question des âges auxquels les hommes étaient mobilisables. L’article de Gilles, Guironnet et Parent utilise les âges légaux de mobilisation mais il y a une proportion importante d’hommes qui n'étaient pas dans cette tranche d'âges mais qui ont quand même combattu, ce qui fausse leur proportion de morts par mobilisables.\footcites[538]{mariot2014} Bien que les deux auteurs apprécient la tentative de modélisation , ils concluent sans ambages à son caractère endémique dans l'historiographie de la Première Guerre mondiale. Elle  manque de données brutes, d’informations sur la manière dont ils sont parvenus à  ses résultats et de réponses aux questions de l'historiographie de la Première Guerre mondiale.\footcites[541]{mariot2014}\\


La dichotomie entre ces deux publications a révélé l'indissociabilité de la méthodologie et de la problématique, et va ainsi me servir de guide pour ma propre étude.  Ma problématique est de faire la lumière sur les différents régiments coloniaux et métropolitains qui ont été amenés à servir pendant la guerre du Rif et de situer les résultats de mon étude dans l’historiographie manquante de ce conflit. 






