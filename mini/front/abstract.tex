\section*{Résumé}
\addcontentsline{toc}{chapter}{Résumé}
 Il existe un manque important dans l'historiographie du Maroc et de l'empire français au sujet du sort des troupes coloniales et metropolitaines pendant la guerre du Rif. L'objectif de ce rapport est de mettre en lumière l'histoire et la trajectoire des differents régiments qui ont participé à cette guerre. Les méthodes quantitatives nous permettent de réaliser une étude statistique ainsi qu'une géographie économique des ces soldats. Nous utilisons des méthodes classiques de sciences des données afin de completer les sources qualitatives dont nous disposons sur cette guerre. 

\medskip

\textbf{Mots-clés: méthodes quantitatives ; géographie économique ; base de données ; armée de terre ; histoire militaire ; histoire coloniale ; humanités numériques ; HTR}

\textbf{Informations bibliographiques:} Theophile Miailhe, \textit{Les soldats d'un conflit oublié : Une étude statistique des morts de la Guerre du Rif: 1925 - 1926}, mémoire de master 1 \og Humanités Numériques\fg{}, dir. [Nicolas Mariot, Julian Randon-Furling], Université Paris, Sciences \& Lettres, 2022.


\section*{Abstract}
\addcontentsline{toc}{chapter}{Abstract}
The historiography of Morocco and France's colonial empire is rather shallow on the subject of the the French army's involvement in the Rif War. The objective of this dissertation is to highlight the diversity of regiments  that participated in the Rif War and the maintenance of France's colonial empire. Quantitative methods allow us to dissect and group soldiers in more precise ways, which would have been time consuming and costly in the past. The marriage of both quantitative and qualitative sources aims to provide new insights on the conflict. 

\medskip

\textbf{Keywords: quantitative history ; databases ; military history ; colonial history ; digital humanities ; HTR}

\textbf{Bibliographic Information:} Theophile Miailhe, \textit{Soldiers of a forgotten war: a quantative approach to the Rif campaign: 1925 - 1926}, M.A. thesis \og Digital Humanities\fg{}, dir. [Nicolas Mariot, Julian Randon-Furling], Université Paris, Sciences \& Lettres, 2022.

\clearpage